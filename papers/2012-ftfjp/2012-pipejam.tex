\documentclass{sig-alternate}

\usepackage{hyperref}

\begin{document}

\title{Verification Games: Making Verification Fun!}

\subtitle{[Project Abstract]}

% \titlenote{A full version of...}}

\numberofauthors{6}
%
\author{
Werner Dietl
\qquad
Michael D. Ernst
\qquad
Nat Mote
\qquad
Brian Walker\\
%
{\normalsize
Programming Languages \& Software Engineering Group,
University of Washington}\\
\url{{wmdietl,mernst,nmote,bdwalker}@cs.washington.edu}
%
\and
%
Seth Cooper
\qquad
Timothy Pavlik
\qquad
Zoran Popovi\'c\\
%
{\normalsize
Center for Game Science, University of Washington}\\
\url{{scooper,zoran}@cs.washington.edu},
\url{timothy.pavlik@gmail.com}
}


%% % 1st. author
%% \alignauthor
%% Ben Trovato\titlenote{Dr.~Trovato insisted his name be first.}\\
%%        \affaddr{Institute for Clarity in Documentation}\\
%%        \affaddr{1932 Wallamaloo Lane}\\
%%        \affaddr{Wallamaloo, New Zealand}\\
%%        \email{trovato@corporation.com}
%% % 2nd. author
%% \alignauthor
%% G.K.M. Tobin\titlenote{The secretary disavows
%% any knowledge of this author's actions.}\\
%%        \affaddr{Institute for Clarity in Documentation}\\
%%        \affaddr{P.O. Box 1212}\\
%%        \affaddr{Dublin, Ohio 43017-6221}\\
%%        \email{webmaster@marysville-ohio.com}
%% % 3rd. author
%% \alignauthor Lars Th{\o}rv{\"a}ld\titlenote{This author is the
%% one who did all the really hard work.}\\
%%        \affaddr{The Th{\o}rv{\"a}ld Group}\\
%%        \affaddr{1 Th{\o}rv{\"a}ld Circle}\\
%%        \affaddr{Hekla, Iceland}\\
%%        \email{larst@affiliation.org}
%% \and  % use '\and' if you need 'another row' of author names
%% % 4th. author
%% \alignauthor Lawrence P. Leipuner\\
%%        \affaddr{Brookhaven Laboratories}\\
%%        \affaddr{Brookhaven National Lab}\\
%%        \affaddr{P.O. Box 5000}\\
%%        \email{lleipuner@researchlabs.org}
%% % 5th. author
%% \alignauthor Sean Fogarty\\
%%        \affaddr{NASA Ames Research Center}\\
%%        \affaddr{Moffett Field}\\
%%        \affaddr{California 94035}\\
%%        \email{fogartys@amesres.org}
%% % 6th. author
%% \alignauthor Charles Palmer\\
%%        \affaddr{Palmer Research Laboratories}\\
%%        \affaddr{8600 Datapoint Drive}\\
%%        \affaddr{San Antonio, Texas 78229}\\
%%        \email{cpalmer@prl.com}
%% }

%% \additionalauthors{Additional authors: John Smith (The Th{\o}rv{\"a}ld Group,
%% email: {\texttt{jsmith@affiliation.org}}) and Julius P.~Kumquat
%% (The Kumquat Consortium, email: {\texttt{jpkumquat@consortium.net}}).}


\date{}

\maketitle

\begin{abstract}
Program verification is the only way to be certain that a given
piece of software is free of (certain types of) errors --- errors that
could otherwise disrupt operations in the field.  To date, formal
verification has been done manually, by specially-trained engineers.  Labor
costs have heretofore made formal verification too costly to apply beyond
small, critical software components.

Our goal is to make verification more cost-effective by reducing the skill
set required for program verification, and increasing the pool of people
capable of performing program verification.  Our approach is to transform
the verification task into a visual puzzle task --- a game --- that
gets solved by people. The solution of the puzzle is then translated
back into a proof of the verification task.
The puzzle is engaging and intuitive enough that ordinary people can through
game-play become experts.  The community of players will grow over
time and ease the learning curve for novices.

In this paper, we present the Verigames project and our
Pipe Jam prototype.  Our system maps source code's typeflow properties
into a network of pipes.
Pipe widths, which are controlled by the player, directly map to type
annotations in programs that can be mechanically checked and provide a
proof of partial correctness.
This general mechanism can cover a wide variety of the most important
security issues.

We believe that humans may have an edge over automated systems in
certain situations, notably when the program is not verifiable.  A
program is unverifiable when it has a bug, or when it is correct for
reasons that are beyond the power of the verification system.  In
either case, the failure is communicated back to an expert, who only
considers cases that the crowd of players cannot resolve.
%
If successful, our approach will reduce the cost of program verification
and enable it to be applied more broadly.


\end{abstract}

% A category with the (minimum) three required fields
% \category{H.4}{Information Systems Applications}{Miscellaneous}
%A category including the fourth, optional field follows...
% \category{D.2.8}{Software Engineering}{Metrics}[complexity measures, performance measures]

% \terms{Theory}

% \keywords{ACM proceedings, \LaTeX, text tagging}

\iffalse

\section{FTfJP 2012 Call for Papers}

From \url{http://www.comp.nus.edu.sg/~ftfjp/}:

\begin{quote}
Contributions (of up to 6 pages in the ACM 2-column style) are sought
on open questions, new developments, or interesting new applications
of formal techniques in the context of Java or similar
languages. Contributions should not merely present completely finished
work, but also raise challenging open problems or propose speculative
new approaches. We particularly welcome contributions that simply
suggest good topics for discussion at the workshop, or raise issues
that you feel deserve the attention of the research community.
\end{quote}

Dates:

abstract submission: 16 March 2012 (anywhere on Earth)\\
full paper submission: 25 March 2012 (anywhere on Earth)\\
notification: 29 April 2012\\
camera-ready paper: TBD\\
conference date: 12 June 2012\\

\fi


\section{Introduction}

\cite{checker-framework-website}


\section{Conclusions}


\section{Acknowledgments}


\bibliographystyle{abbrv}
\bibliography{2012-pipejam,bibstring-abbrev,ernst,invariants,other,types}


\balancecolumns


\end{document}
